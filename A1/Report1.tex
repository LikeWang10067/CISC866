\documentclass{article}
\usepackage{graphicx, listings, xcolor, subcaption, hyperref}
\usepackage[letterpaper,top=1cm,bottom=2cm,left=1cm,right=1cm,marginparwidth=1cm]{geometry}

\lstset{
  basicstyle=\ttfamily\footnotesize,
  keywordstyle=\color{blue},
  stringstyle=\color{red},
  commentstyle=\color{gray},
  breaklines=true,
  showstringspaces=false,
  backgroundcolor=\color{gray!10}
}

\title{CISC866 Lab Report}
\author{Like Wang}

\begin{document}

\maketitle

\section{Encryption Mode: ECB vs. CBC}

\subsection{Display both encrypted image files}

\textit{Note: I chose to use the following method to get rid of the header issue:}
\begin{lstlisting}[language=Python]
$ head -c 54 original1.bmp > header
$ tail -c +55 encrypted1.bmp > body
$ cat header body > new1.bmp
\end{lstlisting}
\textbf{Encryption command for ECB mode} (Note: iv not used by this cipher):
\begin{lstlisting}[language=Python]
$ openssl enc -aes-128-ecb -e -in pic_original1.bmp -out pic_1_ecb_enc.bmp -K 00112233445566778889aabbccddeeff
$ openssl enc -aes-128-ecb -e -in pic_original2.bmp -out pic_2_ecb_enc.bmp -K 00112233445566778889aabbccddeeff
\end{lstlisting}
Picture look like this:
\begin{figure}[h]
    \centering
    \begin{subfigure}{0.45\textwidth}
        \centering
        \includegraphics[height=3cm]{images/pic_1_ecb.png}
        \caption{Encrypted pic original1.bmp}
    \end{subfigure}
    \begin{subfigure}{0.45\textwidth}
        \centering
        \includegraphics[height=3cm]{images/pic_2_ecb.png}
        \caption{Encrypted pic original2.bmp}
    \end{subfigure}
    \caption{ECB Mode Encrypted Images}
\end{figure}
\textbf{Encryption command for CBC mode}:
\begin{lstlisting}[language=Python]
$ openssl enc -aes-128-cbc -e -in pic_original1.bmp -out pic_1_cbc_enc.bmp -K 00112233445566778889aabbccddeeff -iv 0102030405060708090a0b0c0d0e0f10
$ openssl enc -aes-128-cbc -e -in pic_original2.bmp -out pic_2_cbc_enc.bmp -K 00112233445566778889aabbccddeeff -iv 0102030405060708090a0b0c0d0e0f10
\end{lstlisting}
Picture look like this:
\begin{figure}[h]
    \centering
    \begin{subfigure}{0.45\textwidth}
        \centering
        \includegraphics[height=3cm]{images/pic_1_cbc.png}
        \caption{Encrypted pic original1.bmp}
    \end{subfigure}
    \begin{subfigure}{0.45\textwidth}
        \centering
        \includegraphics[height=3cm]{images/pic_2_cbc.png}
        \caption{Encrypted pic original2.bmp}
    \end{subfigure}
    \caption{CBC Mode Encrypted Images}
\end{figure}
\subsection{Explain my observations}
\begin{itemize}
  \item \textbf{ECB mode} is kind of like a invalid encryption for images, because it reveals patterns in the original image. Even though the colors are changed, the shapes and outlines of objects in the image are still visible.
  \item \textbf{CBC mode} is a much better encryption method, at least from the encrypted cbc images, I can't really tell what the original image looks like.
\end{itemize}
Digging on the reason why, I found that: ECB is weak encryption for images because it's encrypted block wise, meaning that identical plaintext blocks are encrypted into identical ciphertext blocks, while CBC mode uses an initialization vector (IV) and chains the encryption of each block to the previous one, which helps to obscure patterns in the original image.\\
Check the following references for detailed information:\\
\url{https://pycryptodome.readthedocs.io/en/latest/src/cipher/classic.html#ecb-mode}\\
\url{https://pycryptodome.readthedocs.io/en/latest/src/cipher/classic.html#cbc-mode}\\
\url{https://www.highgo.ca/2019/08/08/the-difference-in-five-modes-in-the-aes-encryption-algorithm/}

\subsection{Describe the visual differences}
The original image looks very simmilar, but to me, I first noticed that original1.bmp (1.5 MB) has a bigger file size
than original2.bmp (1.1 MB), then, I tried to zoom both picture, and noticed that original1.bmp has a vector background,
meaning that when I zoom in, the background is still clear.\\
Therefore, the color encoding mechanism (for each pixel) of these 2 images might be different. The second image:
original2.bmp might used RGB color encoding, while original1.bmp might used some other color encoding mechanism.
And note that since AES (CBC and ECB) are block ciphers, meaning that they encrypt fixed size blocks of 128 bits,
so, if we are using RGB color encoding, each pixel is represented by 24 bits, so, each block can only contain
5 pixels (5*24=120 bits) and the remaining 8 bits are carried to the next block. After a certain number of blocks,
the reminder bits will accumulate to 24 bits, and form a new pixel. And that's why we see partern like the image
shown below.\\
\begin{figure}[h]
    \centering
    \includegraphics[height=3cm]{images/slash_pattern.png}
    \caption{Zoomed Encrypted pic original2.bmp}
\end{figure}

\section{Padding}
\subsection{Padding for ECB, CBC, CFB, OFB, and CTR modes}
ECB and CBC modes require padding, while CFB, OFB, and CTR modes do not require padding. I was inspired by one the things
Professor mentioned in the lecture: "In CTR modes, we will generate a keystream that is the same length as the plaintext".
I was thinking, that could be the reason why CFB, OFB, and CTR modes do not require padding. I confirmed my guess by
the following references:\\
\url{https://en.wikipedia.org/wiki/Block_cipher_mode_of_operation}\\


\end{document}